\section{Podsumowanie (Maciej Plewka)}
Zrealizowany projekt miał na celu opracowania aplikacji mobilnej, umożliwiającej wykonanie sztuczki karcianej. Wynikiem pracy jest działający program możliwy do uruchomienia na telefonach z systemem Android w wersji 5.0+. W realizacji celu wykorzystano wbudowane komponenty telefonu takie jak aparat, wibrator oraz moduł NFC. Dodatkowo wytworzono własne kaskady klasyfikatorów zdolnych do rozpoznawania obiektów. Zrealizowano obsługę aparatu z poziomu aplikacji, możliwość odczytu i zapisu wiadomości do znaczników NFC, wykorzystanie wibratora do wibrowania w zdefiniowanych przez nas odstępach czasu, detekcję obiektów na wykonanym zdjęciu oraz wytrenowaną własną kaskadę Haar'a.
Dodatkowo można wskazać kilka funkcji, które usprawniłyby naszą aplikację lub rozszerzyły by jej funkcjonalność:
\begin{itemize}
    \item Zastosowanie innych sposobów detekcji, np. przy użyciu splotowych sieci neuronowych.
    \item Umożliwienie wysyłania wykonanego zdjęcia przy użyciu poczty elektronicznej
    \item Zastosowanie wstępnej detekcji w czasie rzeczywistym podczas wykonywania zdjęcia.
    \item Dodanie samouczka, który wyjaśniał by w jaki sposób używać aplikacji
    \item Umieszczenie aplikacji w Sklepie Play
\end{itemize}
