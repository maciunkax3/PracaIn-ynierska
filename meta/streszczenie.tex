\section*{Streszczenie}
Niniejsza praca przedstawia projekt oraz sposób realizacji aplikacji mobilnej. Wykonany program ma być wykorzystywany do wykonania sztuczki karcianej. Na potrzeby triku wykonano aplikację przeznaczoną na smartfony z systemem Android, która wykorzystuje następujące komponenty telefonu: aparat, moduł NFC, wibrator oraz pamięć wewnętrzną telefonu. W pracy przedstawiono podstawy związane z działaniem oraz wykonaniem aplikacji na tej platformie. Jednym z celów pracy była detekcja obiektów znajdujących się na zdjęciu, z tego powodu niezbędne było wykorzystanie wbudowanego aparatu, jako źródło zdjęcia w którym będzie szukany obiekt. Po wykonaniu zdjęcia, obraz zapisywany jest w pamięci telefonu. Innym wariantem sztuczki jest wibrowanie binarnej reprezentacji karty, przy użyciu wibratora. Informacje na temat karty w postaci zrozumiałej dla telefonu umieszczone są w znacznikach NFC umieszczonych w kartach ze zwykłej talii kart do gier. W pracy zawarto szczegóły dotyczące wykorzystanych rozwiązań w celu komunikacji telefonu ze znacznikiem. Detekcja obiektów została zrealizowana przy użyciu kaskadowych klasyfikatorów cech Haar'a. Metoda ta działa na zasadzie dopasowywania charakterystycznych dla obiektu miejsc. W pracy szerzej został opisany cały proces detekcji a także zagadnienia dotyczące wytrenowania nowej kaskady, będącej w stanie rozpoznawać dowolnie wybrany obiekt. W dokumencie zostały również przedstawione narzędzia wykorzystane w procesie treningu kaskad oraz detekcji.


%Niniejsza praca przedstawia projekt oraz sposób realizacji aplikacji mobilnej. Wykonany
%program ma być wykorzystywany do wykonania sztuczki karcianej. Aplikacja jest przeznaczona na
%urządzenia z systemem Android. W pracy przedstawiono podstawy związane z działaniem oraz
%wykonaniem aplikacji na tej platformie. W celu realizacji projektu konieczne było wykorzystanie
%wbudowanych w telefon komponentów. Wykorzystanymi podzespołami były, aparat w celu wyko-
%nania zdjęcia, moduł NFC by odczytywać i zapisywać informacje w znacznikach umieszczonych
%na kartach do gry oraz wibrator potrafiący wibrować binarną reprezentację karty. Dodatkowo
%należało zaimplementować możliwość detekcji obiektów na wykonanym zdjęciu. Do zrealizowania
%detekcji wykorzystano kaskadowe klasyfikatory cech Haar’a. Działanie tego rozwiązania zostało
%opisane w pracy, tak jak proces trenowania własnej kaskady. W następnych rozdziałach opisano
%dokładniej najważniejsze elementy wchodzące w skład aplikacji, a także sposób ich implementacji.
%Wyjaśnione zostały także pojęcia potrzebne do zrozumienia działania poszczególnych procesów
%wykorzystanych w celu wykonania projektu.
