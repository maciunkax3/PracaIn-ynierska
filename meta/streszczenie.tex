\section*{Streszczenie}
Zaprezentowany projekt inżynierski przedstawia projekt oraz sposób realizacji aplikacji mobilnej. Wykonany program ma być wykorzystywany do wykonania sztuczki karcianej. Aplikacja jest przeznaczona na urządzenia z systemem Android. W pracy przedstawiono podstawy związane z działaniem oraz wykonaniem aplikacji na tej platformie. W celu realizacji projektu konieczne było wykorzystanie wbudowanych w telefon komponentów. Wykorzystanymi podzespołami były, aparat w celu wykonania zdjęcia, moduł NFC by odczytywać i zapisywać informacje w znacznikach umieszczonych na kartach do gry oraz wibrator potrafiący wywibrować binarną reprezentację karty. Dodatkowo należało zaimplementować możliwość detekcji obiektów na wykonanym zdjęciu. Do zrealizowania detekcji wykorzystano kaskadowe klasyfikatory cech Haar'a. Działanie tego rozwiązania zostało opisane w pracy, tak jak proces trenowania własnej kaskady. W następnych rozdziałach opisano dokładniej najważniejsze elementy wchodzące w skład aplikacji, a także sposób ich implementacji. Wyjaśnione zostały także pojęcia potrzebne do zrozumienia działania poszczególnych procesów wykorzystanych w celu wykonania projektu. 
