\section*{Abstract}
The following Engineering Thesis presents the project and the method of implementing a mobile application. The prepared program is meant to be used in order to perform card tricks. For the needs of trick, it was made an application, designed for smartphones with Android operating system, which is using built-in components and these are: camera, NFC module, vibrator and phone's local storage. The paper presents the basics of operating and developing an application on such platform. One of the goals of the task was to detect objects located on the pictures, so it was nesseary to use built-in camera, as a source of picture, on which objects will be detected. After taking picture, image is saved in phone’s storage. Another variant of the trick is to vibrating the binary representation of the card. All info needed by phone about card is saved in NFC tags and these are placed in play cards.The work contains details of the solutions used to communicate the telephone with the tag. Object detection was accomplish by using Haar-like cascade classifiers. This method is based on maching characteristics places of the object. In the paper, the entire detection process was described in more details, as well as issues related to training a new cascade, which is able to recognize any choosen object. The document also presents the tools used in the process of training cascades and object detection.



%The following Engineering Thesis presents the project and the method of implementing a mobile application. The %prepared program is meant to be used in order to perform card tricks. The application is dedicated for systems %running the Android operating system. The paper presents the basics of operating and developing an application on %such platform.  In order to implement the project, it was necessary to use the phone's built-in components. The %components used were: the camera for the purpose of taking a picture, the NFC module in order to read and write %information in tags placed on the playing cards, and a vibrator capable of vibrating the binary representation of the %card. In addition, it was necessary to implement the ability to detect objects in the taken picture. Haar-like %cascade classifiers were used to implement the detection. The operation of this solution has been described in the %paper, as well as the process of training one's own cascade. The following chapters describe more specifically the %key elements that make up the application, as well as the method of their implementation. The concepts needed to %understand the operation of individual processes used to implement the project are also explained.
